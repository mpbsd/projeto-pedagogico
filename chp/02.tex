\chapter{Capítulo Dois}
\label{chp:capitulo-dois}

\section{Ementas}
\label{sec:ementas}

\subsection{Matemática Elementar I}
\label{subsec:ementa-matematica-elementa-i}

\begin{enumerate}
  \item Período: 1\textsuperscript{o};
  \item Modalidade: Obrigatória;
  \item Carga Horária:
    \begin{enumerate}
      \item Teórica: 68 horas aula;
      \item Prática: não há;
    \end{enumerate}
  \item Bibliografia:
    \begin{enumerate}
      \item Básica:
        \cite{caracca1951conceitos},
        \cite{iezzi2004fundamentos},
        \cite{lima1997matematica};
      \item Complementar:
        \cite{caracca1951conceitos},
        \cite{iezzi2004fundamentos},
        \cite{lima1997matematica}.
    \end{enumerate}
\end{enumerate}

\begin{ementa}
  Teoria de Conjuntos. Estudo das Equações, Inequações e Funções: Polinomial do
  1\textsuperscript{o} e 2\textsuperscript{o} grau, modular, exponencial e
  logarítmica. Função Composta e Função Inversa. Estudo das sequências
  numéricas: lei de formação de uma sequência numérica, Progressão Aritmética e
  Progressão geométrica e aplicações.
\end{ementa}

% \begin{table}[h]
%   \centering
%   \begin{tabular}{llll}
%     \toprule
%     \multicolumn{4}{c}{Matemática Elementar I} \\
%     \midrule
%     \multicolumn{4}{c}{
%       \begin{minipage}{\textwidth}
%         \vspace{1ex}
%         \small\textit{
%           Teoria de Conjuntos. Estudo das Equações, Inequações e Funções:
%           Polinomial do 1\textsuperscript{o} e 2\textsuperscript{o} grau,
%           modular, exponencial e logarítmica. Função Composta e Função Inversa.
%           Estudo das sequências numéricas: lei de formação de uma sequência
%           numérica, Progressão Aritmética e Progressão geométrica e aplicações.
%         }
%         \vspace{1.5ex}
%       \end{minipage}
%     } \\
%     \midrule
%     \texttt{Período}: \small\textit{1\textsuperscript{o}} &
%     \multirow{2}{*}{\texttt{Carga Horária}}               &
%     \texttt{Teórica}                                      &
%     \small\textit{68 horas aula}                          \\
%     \cmidrule{3-4}
%     \texttt{Modalidade}: \small\textit{Obrigatória} &
%     {}                                              &
%     \texttt{Prática}                                &
%     \small\textit{não há}                           \\
%     \midrule
%     {}                                                                                  &
%     \multirow{2}{*}{\texttt{Bibliografia}}                                              &
%     \texttt{Básica}                                                                     &
%     \cite{caracca1951conceitos}, \cite{iezzi2004fundamentos}, \cite{lima1997matematica} \\
%     \cmidrule{3-4}
%     {}                                                                                  &
%     {}                                                                                  &
%     \texttt{Complementar}                                                               &
%     \cite{caracca1951conceitos}, \cite{iezzi2004fundamentos}, \cite{lima1997matematica} \\
%     \bottomrule
%   \end{tabular}
%   \caption{Ementa para Matemática Elementar I}
%   \label{tbl:ementa-para-matematica-elementar-i}
% \end{table}
